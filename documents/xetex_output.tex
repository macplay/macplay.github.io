%! TEX program = xelatex

\documentclass[a4paper]{article}
\usepackage[top=0.5in,bottom=0.5in,left=0.5in,right=0.5in]{geometry}
% generated by Docutils <http://docutils.sourceforge.net/>
% rubber: set program xelatex
\usepackage{fontspec}
% \defaultfontfeatures{Scale=MatchLowercase}
% straight double quotes (defined T1 but missing in TU):
\ifdefined \UnicodeEncodingName
  \DeclareTextCommand{\textquotedbl}{\UnicodeEncodingName}{%
    {\addfontfeatures{RawFeature=-tlig,Mapping=}\char34}}%
\fi
\usepackage{ifthen}
\usepackage{color}
\usepackage{float} % float configuration
\floatplacement{figure}{H} % place figures here definitely
\usepackage{graphicx}
\setcounter{secnumdepth}{0}

%%% Custom LaTeX preamble
% Linux Libertine (free, wide coverage, not only for Linux)
\setmainfont{Helvetica}
\setsansfont{Arial}
\setmonofont[HyphenChar=None,Scale=MatchLowercase]{Menlo}

\usepackage{xeCJK}
\setCJKmainfont{PingFang SC}
\parindent 2em
\setCJKsansfont{Hiragino Sans GB}
\XeTeXlinebreaklocale "zh"
\XeTeXlinebreakskip = 0pt plus 1pt minus 0.1pt

%%% User specified packages and stylesheets

%%% Fallback definitions for Docutils-specific commands

% class handling for environments (block-level elements)
% \begin{DUclass}{spam} tries \DUCLASSspam and
% \end{DUclass}{spam} tries \endDUCLASSspam
\ifx\DUclass\undefined % poor man's "provideenvironment"
 \newenvironment{DUclass}[1]%
  {\def\DocutilsClassFunctionName{DUCLASS#1}% arg cannot be used in end-part of environment.
     \csname \DocutilsClassFunctionName \endcsname}%
  {\csname end\DocutilsClassFunctionName \endcsname}%
\fi
% basic code highlight:
\providecommand*\DUrolecomment[1]{\textcolor[rgb]{0.40,0.40,0.40}{#1}}
\providecommand*\DUroledeleted[1]{\textcolor[rgb]{0.40,0.40,0.40}{#1}}
\providecommand*\DUrolekeyword[1]{\textbf{#1}}
\providecommand*\DUrolestring[1]{\textit{#1}}
% numeric or symbol footnotes with hyperlinks
\providecommand*{\DUfootnotemark}[3]{%
  \raisebox{1em}{\hypertarget{#1}{}}%
  \hyperlink{#2}{\textsuperscript{#3}}%
}
\providecommand{\DUfootnotetext}[4]{%
  \begingroup%
  \renewcommand{\thefootnote}{%
    \protect\raisebox{1em}{\protect\hypertarget{#1}{}}%
    \protect\hyperlink{#2}{#3}}%
  \footnotetext{#4}%
  \endgroup%
}

% inline markup (custom roles)
% \DUrole{#1}{#2} tries \DUrole#1{#2}
\providecommand*{\DUrole}[2]{%
  % backwards compatibility: try \docutilsrole#1{#2}
  \ifcsname docutilsrole#1\endcsname%
    \csname docutilsrole#1\endcsname{#2}%
  \else
    \csname DUrole#1\endcsname{#2}%
  \fi%
}

% text mode subscript
\ifx\textsubscript\undefined
  \usepackage{fixltx2e} % since 2015 loaded by default
\fi

% titlereference role
\providecommand*{\DUroletitlereference}[1]{\textsl{#1}}

% hyperlinks:
\ifthenelse{\isundefined{\hypersetup}}{
  \usepackage[colorlinks=true,linkcolor=blue,urlcolor=blue]{hyperref}
  \usepackage{bookmark}
  \urlstyle{same} % normal text font (alternatives: tt, rm, sf)
}{}


%%% Body
\begin{document}

% title: 从 Markdown 到 reStructuredText(三)

% slug: cong-markdown-dao-restructuredtextsan

% date: 2017-10-11 15:26:32 UTC+08:00

% tags: markup, reST

% category:

% link:

% description:

% type: text

% nocomments:

% password:

% previewimage:

本文是《从 Markdown 到 reStructuredText》系列文章的第三篇。和 Markdown 一样,reStructuredText 也是一种易读易写的纯文本标记语言,不过功能上更加强大(而且标准统一)。如果想了解其对应于 Markdown 的基本语法,请阅读 \href{../cong-markdown-dao-restructuredtext/}{第一篇文章} 。本文继续 \href{../cong-markdown-dao-restructuredtexter/}{上一篇文章} 的话题,聊一聊标记语言的样式问题,确切的说是 reStructuredText 在静态博客 Nikola 中的样式写法。

“样式?标记语言还需要考虑样式?”估计不少人心里会犯嘀咕。然而,标记语言一直强调的是 \textbf{易读易写} ,无需专门的商业版权软件来编辑, \textbf{同时,纯文本可被转换为其它格式的文档}\DUfootnotemark{id3}{id4}{1} 。转换为其它格式多半少不了样式,样式本身有助于读者对文档内容的理解,传递赏心悦目的文档对读者本人也是一种尊重。标记语言宣称的“毋需关注样式”更多的是谴责 \textbf{过分关注样式} ,以至于丢掉了文档之魂——内容。

个人以为,标记语言自始至终贯彻的原则只有一点: \textbf{内容与样式分离} 。摆脱束缚用最简单的纯文本书写,而必要时又可以套用现成样式模板,导出或专业或活波……不同风格的文档方便分发。
%
\DUfootnotetext{id4}{id3}{1}{%
见 \href{../cong-markdown-dao-restructuredtext/}{第一篇文章} 宗旨 -> reStructuredText 的预期目标
}

% TEASER_END

\phantomsection\label{id5}
\pdfbookmark[1]{文章目录}{id5}
\renewcommand{\contentsname}{文章目录}
\tableofcontents


\section{从图片居中说起%
  \label{id6}%
}

相信有不少人第一次使用 Markdown 书写文章,或者安利别人使用标记语言的时候,都会寻思或者被问到一个问题:“这个预览文章中的图片怎么不居中显示呢?”常年浸泡在多媒体中的普罗大众们,对默认居左显示的“简陋”的图片已经无法忍受了。更不用说,这些图片无法自动缩放,有时会占用比预期中大得多的篇幅。这个问题困扰了相当一部分人(包括笔者在内),经过一番搜索后个人认为找到了原因:Markdown 的大部分解析器生成 HTML 文档时,仅仅包含了基本的标签元素。

查阅 CSS 语法我们发现,要想对图片应用样式则必须通过 \DUroletitlereference{选择器} ,当然你可以直接使用 \DUroletitlereference{元素选择器} img。但这种做法有很大的弊端:当我们使用 Markdown 写静态博客时,其中包含的一些小图片(尤其是行内图片),你肯定不希望它们也居中显示。我们真正需要的是 \DUroletitlereference{类选择器} ,标签属性 \DUroletitlereference{类(class)} 在一定程度上说明了该图片/段落承担的“角色”,是对全文理解非常重要(important),错过该步骤会导致严重后果的信息?还是无关紧要仅仅是锦上添花的提醒(tip)?通过 \DUroletitlereference{类选择器} 就可以对 HTML 文档进行样式调整,实现“图片居中”等效果。

当然笔者最后还是找到了 Markdown 下的解决方案:可以自定义标签属性 \DUroletitlereference{类(class)} 的扩展 \href{https://pythonhosted.org/Markdown/extensions/attr_list.html}{Attribute Lists}\DUfootnotemark{id7}{id8}{2} 。然后一直使用包含该扩展的 Markdown…… 直到遇见 reStructuredText 标记语言。reStructuredText 语法也仅生成基本的 HTML 标签,但还有部分会附加上 \DUroletitlereference{类(class)} ,再加上 reStructuredText 本身语法繁多(譬如图片就有 \DUroletitlereference{image} 和 \DUroletitlereference{figure} 两种),对于大多数人来说已经够用。此外,reStructuredText 毋需扩展提供了自定义 \DUroletitlereference{类(class)} 的方法:其绝大部分 \DUroletitlereference{指令(Directives)} 拥有选项 \DUroletitlereference{class} ;非 \DUroletitlereference{指令(Directives)} 没有选项可填写的,则有专门的 \DUroletitlereference{指令(Directives)} : \DUroletitlereference{class} 让用户自定义。
%
\DUfootnotetext{id8}{id7}{2}{%
其实 Markdown 扩展 \href{https://pythonhosted.org/Markdown/extensions/attr_list.html}{Attribute Lists} 功能也很强大,支持自定义标签 \DUroletitlereference{class} 、 \DUroletitlereference{id} 及其它属性等。不过本文重点在 reStructuredText,关于 Markdown 的一些扩展后面将会有专门的文章谈到。
}


\section{Class 属性的花花世界%
  \label{class}%
}

OK,说了这么多,让我们来看一下引入 \DUroletitlereference{class} 后可以对文档样式做哪些修改。前面已经说过,直接套用现成样式是快速简洁的最佳实践,对静态博客来说,现成样式最好是 \href{http://getbootstrap.com}{Bootstrap} 、 \href{http://materializecss.com}{Materialize} 、 \href{https://semantic-ui.com}{Semantic UI} 等流行的 CSS/JavaScript 框架。这样,只需给元素指定 \DUroletitlereference{class} 即可获得样式,框架中没有包含的样式则需写少量的 CSS 代码。

以下例子均以本博客使用的 \href{https://semantic-ui.com}{Semantic UI} 框架为例。


\subsection{样式套用%
  \label{id9}%
}

\begin{enumerate}
\item \textbf{居中对齐的堆叠段落}

\begin{DUclass}{code}
\begin{DUclass}{rst}
\begin{quote}
{\ttfamily \raggedright \noindent
\DUrole{punctuation}{..}~\DUrole{operator}{\DUrole{word}{class}}\DUrole{punctuation}{::}~ui~center~aligned~stacked~segment\\
~\\
游览瑞典北部的萨勒克国家公园,你可能需要花一些时间在广袤的拉帕谷。这片荒野经常被雨水浸透,山谷被高耸的山峰包围着,萨勒克国家公园是数百座山的家园,其中包括一些瑞典最高的山峰。而第一次来到山谷的游客可以安排一个向导来陪同,因为在拉帕谷没有指定的步道,所以请个导游也是非常必要的。
}
\end{quote}
\end{DUclass}
\end{DUclass}

\DUrole{ui}{\DUrole{center}{\DUrole{aligned}{\DUrole{stacked}{\DUrole{segment}{游览瑞典北部的萨勒克国家公园,你可能需要花一些时间在广袤的拉帕谷。这片荒野经常被雨水浸透,山谷被高耸的山峰包围着,萨勒克国家公园是数百座山的家园,其中包括一些瑞典最高的山峰。而第一次来到山谷的游客可以安排一个向导来陪同,因为在拉帕谷没有指定的步道,所以请个导游也是非常必要的。}}}}}

\item \textbf{带指向的基本蓝色标签}

\begin{DUclass}{code}
\begin{DUclass}{rst}
\begin{quote}
{\ttfamily \raggedright \noindent
\DUrole{punctuation}{..}~\DUrole{operator}{\DUrole{word}{class}}\DUrole{punctuation}{::}~ui~right~pointing~blue~basic~label\\
~\\
\DUrole{literal}{\DUrole{string}{`Semantic~UI`\_}}
}
\end{quote}
\end{DUclass}
\end{DUclass}

\DUrole{ui}{\DUrole{right}{\DUrole{pointing}{\DUrole{blue}{\DUrole{basic}{\DUrole{label}{\href{https://semantic-ui.com}{Semantic UI}}}}}}}

\item \textbf{Small-size 的圆形居中图片}

\begin{DUclass}{code}
\begin{DUclass}{rst}
\begin{quote}
{\ttfamily \raggedright \noindent
\DUrole{punctuation}{..}~\DUrole{operator}{\DUrole{word}{image}}\DUrole{punctuation}{::}~/images/avatar.png\\
~~~\DUrole{name}{\DUrole{class}{:class:}}~\DUrole{name}{\DUrole{function}{ui~centered~small~circular~image}}
}
\end{quote}
\end{DUclass}
\end{DUclass}

\begin{figure}
\begin{center}
    \includegraphics[scale=0.5]{images/avatar.png}
\end{center}
\end{figure}

\item \textbf{直观明了的情绪化消息}

\begin{DUclass}{code}
\begin{DUclass}{rst}
\begin{quote}
{\ttfamily \raggedright \noindent
\DUrole{punctuation}{..}~\DUrole{operator}{\DUrole{word}{class}}\DUrole{punctuation}{::}~ui~info~message\\
~\\
你知道~Admonitions(告诫)也有类似的效果嘛?\\
~\\
\DUrole{punctuation}{..}~\DUrole{operator}{\DUrole{word}{class}}\DUrole{punctuation}{::}~ui~negative~message\\
~\\
非常抱歉我们不能给您申请折扣!
}
\end{quote}
\end{DUclass}
\end{DUclass}

\DUrole{ui}{\DUrole{info}{\DUrole{message}{你知道 Admonitions(告诫)也有类似的效果嘛?}}}

\DUrole{ui}{\DUrole{negative}{\DUrole{message}{非常抱歉我们不能给您申请折扣!}}}
\end{enumerate}

除此之外, \href{https://semantic-ui.com}{Semantic UI} 还有很多可以套用的样式,读者们感兴趣请自行去探索发现。


\subsection{Strikethrough(删除线)%
  \label{strikethrough}%
}

删除线算是写作过程中一项常见的需求。遗憾的是,reStructuredText 并不提供相应的语法。

如果你稍微懂点 CSS 的话,直接在样式表里加一行就可以:

\begin{DUclass}{code}
\begin{DUclass}{css}
\begin{quote}
{\ttfamily \raggedright \noindent
\DUrole{punctuation}{.}\DUrole{name}{\DUrole{class}{strike}}~\DUrole{punctuation}{\{}\DUrole{keyword}{text-decoration}\DUrole{punctuation}{:}~\DUrole{keyword}{\DUrole{constant}{line-through}}\DUrole{punctuation}{;\}}
}
\end{quote}
\end{DUclass}
\end{DUclass}

然后写文章的时候,段落套用已定义好的 \DUroletitlereference{strike} class。

\begin{DUclass}{code}
\begin{DUclass}{rst}
\begin{quote}
{\ttfamily \raggedright \noindent
\DUrole{punctuation}{..}~\DUrole{operator}{\DUrole{word}{class}}\DUrole{punctuation}{::}~strike\\
~\\
我所说的都是错的,包括这一句。
}
\end{quote}
\end{DUclass}
\end{DUclass}

个人后来又加了两行样式,于是就变成这样:

\DUrole{strike}{我所说的都是错的,包括这一句。}

同理,想要修订效果也可以:

\DUrole{amend}{子非鱼,安知鱼之乐耶?}


\subsection{Columns(分栏)%
  \label{columns}%
}

主流 CSS/JavaScript 框架均支持网格系统,就是俗称的“分栏”。 \href{https://semantic-ui.com}{Semantic UI} 的默认网格是 16 个,我们依旧以其为例,其它框架写法可能略有不同。

\begin{DUclass}{code}
\begin{DUclass}{rst}
\begin{quote}
{\ttfamily \raggedright \noindent
\DUrole{punctuation}{..}~\DUrole{operator}{\DUrole{word}{container}}\DUrole{punctuation}{::}~ui~stackable~grid\\
~\\
\DUrole{punctuation}{~~~..}~\DUrole{operator}{\DUrole{word}{class}}\DUrole{punctuation}{::}~four~wide~column\\
~\\
~~~文章段落\\
~\\
\DUrole{punctuation}{~~~..}~\DUrole{operator}{\DUrole{word}{class}}\DUrole{punctuation}{::}~eight~wide~column\\
~\\
~~~文章段落\\
~\\
\DUrole{punctuation}{~~~..}~\DUrole{operator}{\DUrole{word}{class}}\DUrole{punctuation}{::}~four~wide~column\\
~\\
~~~文章段落
}
\end{quote}
\end{DUclass}
\end{DUclass}

渲染结果:

\begin{DUclass}{ui}
\begin{DUclass}{stackable}
\begin{DUclass}{grid}

\DUrole{four}{\DUrole{wide}{\DUrole{column}{阿比斯库国家公园是瑞典的一个占地 77 平方公里的国家公园,这里的自然风光迷人,聚集着北欧的野生动物,适合冬季冒险和夏季远足,也适合欣赏极光和午夜的太阳。冬天的阿比斯库就像极北之地,尤其是在国家公园里狗拉雪橇,棒呆了!而且公园还是看极光的最佳地点之一。}}}

\DUrole{eight}{\DUrole{wide}{\DUrole{column}{皇后岛宫,又名卓宁霍姆宫,是瑞典王室的私人宫殿,于 1991 年被列入联合国教科文组织世界遗产名录。这座建于十七世纪的城堡不仅是当今保存最完好的皇家宫殿,也是全欧洲宫廷建筑中最具代表性的一座。再加上异国风情的中国宫殿、宫廷剧院和华丽的宫廷花园,更为皇城打造出无与伦比的整体感。}}}

\DUrole{four}{\DUrole{wide}{\DUrole{column}{瓦萨沉船博物馆位于动物园岛上,主要展示沉船瓦萨号——世界上唯一保存完好的17世纪沉船。为了提防邻国的侵袭,古斯塔夫二世下令建造了这座瓦萨号战船。这座海事博物馆是斯堪的纳维亚地区最受欢迎的博物馆之一。博物馆中所有珍贵的藏品都是从海底打捞上来的。}}}
\end{DUclass}
\end{DUclass}
\end{DUclass}

其中 \DUroletitlereference{container} 指令用来生成 div 块,其参数用来指定 class,而 \DUroletitlereference{stackable} 意思是让网格支持响应式布局,在屏幕较窄的时候堆叠成一列显示,而不是保持固定的三栏(当然你也可以这么做)。是不是感觉棒呆了? :)


\subsection{加一点点 JavaScript%
  \label{javascript}%
}

注意到一些 JavaScript 插件也是利用 \DUroletitlereference{class} 定位的,那我们在博客模板头文件中引入 JavaScript 库后,再加上几行代码就可以实现一些互动效果。比如之前 \href{../cong-markdown-dao-restructuredtext/}{介绍 reStructuredText 文章} 中点击缩略图加载原图,就是 \DUroletitlereference{class} + JavaScript 的典型应用。博客模板文件里其实只加了三行代码:

\begin{DUclass}{code}
\begin{DUclass}{html}
\begin{quote}
{\ttfamily \raggedright \noindent
\DUrole{punctuation}{<}\DUrole{name}{\DUrole{tag}{script}}\DUrole{punctuation}{>}~\\
~~~~\DUrole{name}{\DUrole{other}{\$}}\DUrole{punctuation}{(}\DUrole{name}{\DUrole{builtin}{document}}\DUrole{punctuation}{).}\DUrole{name}{\DUrole{other}{ready}}\DUrole{punctuation}{(}\DUrole{keyword}{\DUrole{declaration}{function}}\DUrole{punctuation}{()\{}~\\
~~~~~~~~\DUrole{name}{\DUrole{other}{\$}}\DUrole{punctuation}{(}\DUrole{literal}{\DUrole{string}{\DUrole{single}{'.image-reference'}}}\DUrole{punctuation}{).}\DUrole{name}{\DUrole{other}{fancybox}}\DUrole{punctuation}{();}~\\
~~~~\DUrole{punctuation}{\});}~\\
\DUrole{punctuation}{</}\DUrole{name}{\DUrole{tag}{script}}\DUrole{punctuation}{>}
}
\end{quote}
\end{DUclass}
\end{DUclass}

至于为何是 \DUroletitlereference{image-reference} ,相信读者朋友们已经猜到了——不管是 \DUroletitlereference{image} 还是 \DUroletitlereference{figure} 指令,默认附加的 \DUroletitlereference{class} 就是 \DUroletitlereference{image-reference} ,这样写作插入图片的时候就不必自定义 class 了。

\begin{DUclass}{code}
\begin{DUclass}{rst}
\begin{quote}
{\ttfamily \raggedright \noindent
\DUrole{punctuation}{..}~\DUrole{operator}{\DUrole{word}{figure}}\DUrole{punctuation}{::}~/images/icarus.thumbnail.jpg\\
~~~\DUrole{name}{\DUrole{class}{:align:}}~\DUrole{name}{\DUrole{function}{center}}~\\
~~~\DUrole{name}{\DUrole{class}{:target:}}~\DUrole{name}{\DUrole{function}{/images/icarus.jpg}}
}
\end{quote}
\end{DUclass}
\end{DUclass}

渲染结果:

\begin{figure}
\begin{center}
    {\includegraphics[scale=0.25]{images/icarus.jpg}}
\end{center}
\end{figure}

这只是最简单的一个例子。如果觉得有必要的话,还可以继续引入其它 JavaScript 库,比如笔者最近看到的 \href{https://mermaidjs.github.io/}{mermaid} 。书写的时候也简单:就像所有正常段落一样写图表的描述,写完指定段落 \DUroletitlereference{class} 为 \DUroletitlereference{mermaid} 即可。


\section{行内样式%
  \label{id10}%
}

可能读者们已经注意到了,以上样式举例都是针对区块元素。如果想使用行内样式应该怎么办呢?答案是: 使用 \DUroletitlereference{role(角色)} 指令。 \DUroletitlereference{role} 对于行内内容和样式具有非常重要的作用,你可以使用内置的 \DUroletitlereference{role} ,也可以自定义 \DUroletitlereference{role} 。我们其实在 \href{../cong-markdown-dao-restructuredtext/}{第一篇文章} 行内数学公式的语法中已经使用过内置 \DUroletitlereference{role} —— \DUroletitlereference{math} 就是一个内置的 \DUroletitlereference{role} 。对于内置 \DUroletitlereference{role} 我们毋需再定义,直接使用即可。内置 \DUroletitlereference{role} 中比较重要的还有 \DUroletitlereference{sup} 和 \DUroletitlereference{sub} ,即上标下标。


\subsection{上标和下标%
  \label{id11}%
}

直接贴语法用例:

\begin{DUclass}{code}
\begin{DUclass}{rst}
\begin{quote}
{\ttfamily \raggedright \noindent
水分子式为:H\textbackslash{}~\DUrole{name}{\DUrole{attribute}{:sub:}}\DUrole{name}{\DUrole{variable}{`2`}}\textbackslash{}~O;质能方程式:E~=~mc\textbackslash{}~\DUrole{name}{\DUrole{attribute}{:sup:}}\DUrole{name}{\DUrole{variable}{`2`}};谁知道中文~\DUrole{name}{\DUrole{attribute}{:sup:}}\DUrole{name}{\DUrole{variable}{`啥时候`}}~使用~\DUrole{name}{\DUrole{attribute}{:sub:}}\DUrole{name}{\DUrole{variable}{`上下标`}}?
}
\end{quote}
\end{DUclass}
\end{DUclass}

水分子式为:H\textsubscript{2}O;质能方程式:E = mc\textsuperscript{2} ;谁知道中文 \textsuperscript{啥时候} 使用 \textsubscript{上下标} ?

\DUroletitlereference{role} 的使用语法形如: \texttt{:角色名:`内容`} ,上例中 \texttt{\textbackslash{}} 是利用转义符号剔除公式字母间的空隙。


\subsection{行内 HTML%
  \label{html}%
}

reStructuredText 行内 HTML 的书写,则是利用一个特殊的内置 \DUroletitlereference{role} —— \DUroletitlereference{raw} ,并添加相应选项。

\begin{DUclass}{code}
\begin{DUclass}{rst}
\begin{quote}
{\ttfamily \raggedright \noindent
\DUrole{punctuation}{..}~\DUrole{operator}{\DUrole{word}{role}}\DUrole{punctuation}{::}~raw-html(raw)\\
~~~\DUrole{name}{\DUrole{class}{:format:}}~\DUrole{name}{\DUrole{function}{html}}
}
\end{quote}
\end{DUclass}
\end{DUclass}

然后我们就可以使用 \DUroletitlereference{raw-html} 了。比如强制断行:

\begin{DUclass}{code}
\begin{DUclass}{rst}
\begin{quote}
{\ttfamily \raggedright \noindent
\DUrole{name}{\DUrole{attribute}{:raw-html:}}\DUrole{name}{\DUrole{variable}{`<br~/>`}}
}
\end{quote}
\end{DUclass}
\end{DUclass}

比如插入 \href{https://semantic-ui.com}{Semantic UI} 的内置图标:

\begin{DUclass}{code}
\begin{DUclass}{rst}
\begin{quote}
{\ttfamily \raggedright \noindent
\DUrole{name}{\DUrole{attribute}{:raw-html:}}\DUrole{name}{\DUrole{variable}{`<i~class=\textquotedbl{}huge~mail~icon\textquotedbl{}></i>`}}
}
\end{quote}
\end{DUclass}
\end{DUclass}



结合 \hyperref[id9]{样式套用} 做一个漂亮的按钮:

\begin{DUclass}{code}
\begin{DUclass}{rst}
\begin{quote}
{\ttfamily \raggedright \noindent
\DUrole{punctuation}{..}~\DUrole{operator}{\DUrole{word}{class}}\DUrole{punctuation}{::}~ui~blue~button\\
~\\
\DUrole{name}{\DUrole{attribute}{:raw-html:}}\DUrole{name}{\DUrole{variable}{`<i~class=\textquotedbl{}mail~icon\textquotedbl{}></i>`}}~邮件
}
\end{quote}
\end{DUclass}
\end{DUclass}

\DUrole{ui}{\DUrole{blue}{\DUrole{button}{ 邮件}}}


\subsection{自定义 role%
  \label{role}%
}

你也可以自定义 role,当然为了让这些 role 在网页中有意义(行内样式),我们最好事先给它们添加 CSS 样式。以上文中定义过的 \DUroletitlereference{strike} 和 \DUroletitlereference{amend} 来举例:

\begin{DUclass}{code}
\begin{DUclass}{rst}
\begin{quote}
{\ttfamily \raggedright \noindent
\DUrole{punctuation}{..}~\DUrole{operator}{\DUrole{word}{role}}\DUrole{punctuation}{::}~strike\\
\DUrole{punctuation}{..}~\DUrole{operator}{\DUrole{word}{role}}\DUrole{punctuation}{::}~amend
}
\end{quote}
\end{DUclass}
\end{DUclass}

然后就可以在文档中引用了:

\begin{DUclass}{code}
\begin{DUclass}{rst}
\begin{quote}
{\ttfamily \raggedright \noindent
鸟宿池边树,僧~\DUrole{name}{\DUrole{attribute}{:strike:}}\DUrole{name}{\DUrole{variable}{`敲`}}~\DUrole{name}{\DUrole{attribute}{:amend:}}\DUrole{name}{\DUrole{variable}{`推`}}~月下门。
}
\end{quote}
\end{DUclass}
\end{DUclass}

鸟宿池边树,僧 \DUrole{strike}{敲} \DUrole{amend}{推} 月下门。

添加行内按钮也是可以的:

\begin{DUclass}{code}
\begin{DUclass}{rst}
\begin{quote}
{\ttfamily \raggedright \noindent
\DUrole{punctuation}{..}~\DUrole{operator}{\DUrole{word}{role}}\DUrole{punctuation}{::}~ibtn\\
~~~\DUrole{name}{\DUrole{class}{:class:}}~\DUrole{name}{\DUrole{function}{ui~mini~basic~green~button}}~\\
~\\
点击~\DUrole{name}{\DUrole{attribute}{:ibtn:}}\DUrole{name}{\DUrole{variable}{`阅读`}}~继续。
}
\end{quote}
\end{DUclass}
\end{DUclass}

点击 \DUrole{ui}{\DUrole{mini}{\DUrole{basic}{\DUrole{green}{\DUrole{button}{阅读}}}}} 继续。


\section{结语%
  \label{id12}%
}

尽管使用 reStructuredText 意味着更加关注文章内容,但这并不意味着当有样式需求的时候,无法得到满足。本文从文档写作需求出发,基本上全面阐述了应用样式的思路和经验,包含文章分栏、图片居中、删除线、上标下标等等。如果你有更好的静态博客写作实践,欢迎在评论区进行讨论。下一篇文章将会试用和探讨 reStructuredText 的文档导出功能,敬请期待~

\end{document}
